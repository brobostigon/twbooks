\documentclass[10pt,a4paper]{book} 
\usepackage[utf8]{inputenc} 
\usepackage{amsmath} 
\usepackage{amsfonts} 
\usepackage{amssymb} 
\usepackage{graphicx} 
\usepackage{textcomp}
\usepackage{geometry}
 \geometry{
 a4paper,
 total={170mm,257mm},
 left=30mm,
 right=30mm,
 bottom=30mm,
 top=30mm,
 }
\date{17-10-2018} 
\author{Philip Martin Taylor} 
\title{Position Statement} 
\begin{document} 
\begin{flushright} 
  \textbf{Case NO: OX18P00494}
\end{flushright}
\begin{flushright}
  \textbf{In Oxford Family Court on the matter of Ava Diane Briggs(born on 19/07/2018)}
\end{flushright} 
\begin{flushleft}
  Between Philip Martin Taylor(applicant) and Kate Louise Briggs(respondent)
\end{flushleft} 
\begin{flushleft}
  Position Statement of the applicant Philip Martin Taylor, for the directions hearing on the 14Th of November 2018 at 10am, before District Judge Wakem, concerning my Declaration of Parentage application, so i can be part of my daughter Ava's life and be her father, and Ava having her father in her life.
\end{flushleft}
\begin{flushleft}
  \textbf{My position is as follows:}
\end{flushleft}
\begin{enumerate}
  \item I filed decleration of parentage to establish that i am Ava's father, and following that arrangements be made so Ava can have her father in her life, including contact and other arrangements, as well as arrangements concerning when Kate gets ill due to her bipolar disorder, including applying for decleration of parentage so i am recognised as Ava's father.
  \item The only recent contact i have had with Kate, is when Kate called me on the 6Th of October 2018 at roughly 11:30 am regarding her receipt of the decleration of parentage application. In January 2018 when she sent me the first scans of Ava from the twelve week scan, and when i invited Kate to mediation in march 2018, which she refused to attend. And have had no contact at all with my daughter Ava, contact with Ava which Kate has refused me.
  \item I ask that the court, makes a decision as to if i am Ava's father, as well as arrangements so Ava can have me, her father in her life, as well as my daughter Ava in my life.
  \item I also ask the court to make an interim contact arrangement, so my daughter Ava can see me and have contact with me, until a more permanent decision is made at a hearing in the future.
  \item It is also important for Ava's welfare, that arrangements be made that when Kate is ill due to her bipolar disorder. Firstly, when Kate does get ill, or is showing signs of getting ill, steps can be taken to ensure Ava's welfare.
    \item Regarding contact arrangements, and considering that me and Kate dont live far from eachother, about 10 mins driving distance away, it would be good to able to see Ava one day during the week, as well as one day during the weekend, and if possible more often than that. Also to take into account, is because i have limited child care experience, that contact can be with Ava's grandparents, ie my parents, as they have extensive child care experience and can teach me, and then in time that contact being without them. 
    \end{enumerate}
    \begin{flushleft}
      I declare that the contents of this statement are true and accurate to the best of my knowledge and belief.
\end{flushleft}
\begin{center}
  Philip Martin Taylor. 17/10/2018
\end{center}
\end{document}
